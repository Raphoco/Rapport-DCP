\documentclass[12pt]{article}
%\usepackage{natbib}
\usepackage[french]{babel}
\usepackage{url}
\usepackage[utf8x]{inputenc}
\usepackage{graphicx}
\graphicspath{{images/}}
\usepackage{parskip}
\usepackage{fancyhdr}
\usepackage{vmargin}
\usepackage{xcolor}
\usepackage{bbm}
\usepackage{amsmath,amssymb}
\usepackage{amsthm}
\usepackage{dsfont}
\usepackage{stmaryrd}
\usepackage{systeme}
\usepackage{enumitem}
\usepackage{xcolor}
\usepackage{pifont}
\usepackage[cache=false]{minted}
\definecolor{LightGray}{gray}{0.95}


\title{Rapport}
\author{PIARD A. - JACQUET R.}
\date{\today}

\makeatletter
\let\thetitle\@title
\let\theauthor\@author
\let\thedate\@date
\makeatother

\pagestyle{fancy}
\fancyhf{}
\rhead{\theauthor}
\lhead{\thetitle}
\cfoot{\thepage}
\def\dotfill#1{\cleaders\hbox to #1{.}\hfill}
\newcommand\dotline[2][.5em]{\leavevmode\hbox to #2{\dotfill{#1}\hfil}}

\begin{document}

\begin{titlepage}
	\centering
    \vspace*{0.5 cm}
    \textsc{\LARGE Droit et Conduite de Projet . 2020-2021}\\[1.0 cm]
    \dotline[15pt]{15cm}\\
	\includegraphics[scale = 2.2]{logo.png}
	\dotline[15pt]{15cm}\\
	\vspace{1.5cm}
	\textsc{\Large Faculté des Sciences et Techniques}\\
	\textsc{\large Master 1 - Maths. CRYPTIS}\\[1.0 cm]
	\rule{\linewidth}{0.2 mm} \\[0.4 cm]
	{ \huge \bfseries \color{blue} \thetitle}\\
	\rule{\linewidth}{0.2 mm} \\[1.5 cm]
	
	\begin{minipage}{0.4\textwidth}
		\begin{flushleft} \large
			\emph{A l'attention de :}\\
			M. CRESPIN\\
			M. DUSART\\
			M. CONCHON\\
		\end{flushleft}
	\end{minipage}
	\begin{minipage}{0.5\textwidth}
    	\begin{flushright} \large
		\emph{Rédigé par :}\\
		PIARD A.\\
		JACQUET R.\\
		\phantom{a}\\
		\end{flushright}
	\end{minipage}\\[2 cm]
\end{titlepage}

\tableofcontents
\pagebreak

\section{Introduction}
Le but de notre projet était d'entraîner et de tester les compétences, en terme de programmation, de nos collègues du \textsf{Master 1 Maths CRYPTIS}. Comme énoncé dans notre fiche d'avant-projet, deux \textsl{Competitive Programming Test} étaient prévus, avec les quatres mêmes participants pour chacun d'eux, et nous avions deux protocoles organisationnelles qui s'offraient à nous en vue de la condition sanitaire actuelle. Le protocole ayant été retenu est celui du distanciel.\\
\\
Le premier sujet a été traité le 1er février 2021 et le second le 8 février 2021.

\vfill \eject

\section{Premières réactions}
À l'issue du premier \textsl{Competitive Programming Test}, nous nous sommes rendus compte que les étudiants ont mis beaucoup plus de temps que prévu sur les exercices, ce qui fait qu'aucun d'eux n'a terminé le premier sujet. Il se trouve que les exercices n'étaient pas triés par ordre de difficulté mais qu'ils ont tous choisis de commencer par le premier exercice.\\\\
Nous les avons légèrement questionnés et aidés à comprendre ce qui n'allait pas, en vue du second \textsl{Competitive Programming Test} qui était prévu la semaine suivante. Celui-ci s'est bien mieux passé. Les étudiants n'ont pas terminés le sujet mais ils ont tous traités au moins un exercice. La plupart des étudiants s'étant intéressés à deux exercices.

\section{Analyses des sujets}
\subsection{Sujet n°1}
%Joindre image du sujet
%Dire ce qui semble les avoir gênés, nos réponses apportées, etc

\subsection{Sujet n°2}
%Joindre image du sujet
%Dire ce qui semble les avoir gênés, nos réponses apportées, etc

\section{Comparaison sujet n°1 - sujet n°2 par rapport à chaque étudiant}

\section{Comparaisons entre étudiants}
\subsection{Sujet n°1}

\subsection{Sujet n°2}

\end{document}